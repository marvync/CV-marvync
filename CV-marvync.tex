%-------------------------------------
% LaTeX Resume for Physicist
% Author : Marvyn Inga
%-------------------------------------

\documentclass[letterpaper, 12pt]{article}[leftmargin=*]

\usepackage[empty]{fullpage}
\usepackage{enumitem}
\usepackage{ifxetex}
\ifxetex
  \usepackage{fontspec}
  \usepackage[xetex]{hyperref}
\else
  \usepackage[utf8]{inputenc}
  \usepackage[T1]{fontenc}
  \usepackage[pdftex]{hyperref}
\fi
\usepackage{fontawesome}
\usepackage[sfdefault,light]{FiraSans}
\usepackage{anyfontsize}
\usepackage{xcolor}
\usepackage{tabularx}
\usepackage{pdflscape}
\usepackage{mhchem}

%-------------------------------------------------- SETTINGS HERE --------------------------------------------------
% Header settings
\def \fullname {Marvyn Inga}
\def \subtitle {Physicist, \faMale}

\def \linkedinicon {\faLinkedin}
\def \linkedinlink {https://www.linkedin.com/in/marvync/}
\def \linkedintext {/marvync}

\def \phoneicon {\faPhone}
\def \phonetext {+55-21-975106507}

\def \emailicon {\faEnvelope}
\def \emaillink {mailto:marvyn.inga@gmail.com}
\def \emailtext {marvyn.inga@gmail.com}

\def \githubicon {\faGithub}
\def \githublink {https://github.com/marvync}
\def \githubtext {/marvync}

\def \headertype {\doublecol} % \singlecol or \doublecol

% Misc settings
\def \bulletstyle {\faAngleRight}

% Define colours
\definecolor{primary}{HTML}{000000}
\definecolor{secondary}{HTML}{00539C}
\definecolor{accent}{HTML}{263238}
\definecolor{links}{HTML}{1565C0}

%------------------------------------------------------------------------------------------------------------------- 

% Defines to make listing easier
\def \linkedin {\linkedinicon \hspace{3pt}\href{\linkedinlink}{\linkedintext}}
\def \phone {\phoneicon \hspace{3pt}{ \phonetext}}
\def \email {\emailicon \hspace{3pt}\href{\emaillink}{\emailtext}}
\def \github {\githubicon \hspace{3pt}\href{\githublink}{\githubtext}}
\def \website {\websiteicon \hspace{3pt}\href{\websitelink}{\websitetext}}

% Adjust margins
\addtolength{\oddsidemargin}{-0.55in}
\addtolength{\evensidemargin}{-0.55in}
\addtolength{\textwidth}{1.1in}
\addtolength{\topmargin}{-0.6in}
\addtolength{\textheight}{1.1in}

% Define the link colours
\hypersetup{
    colorlinks=true,
    urlcolor=links,
}

% Set the margin alignment 
\raggedbottom
\raggedright
\setlength{\tabcolsep}{0in}

%-------------------------
% Custom commands

% Sections
\renewcommand{\section}[2]{
  \colorbox{secondary}{\color{white}\raggedbottom\normalsize\textbf{{#1}{\hspace{7pt}#2}}}
}

% Entry start and end, for spacing
\newcommand{\resumeEntryStart}{\begin{itemize}[leftmargin=2.5mm]\itemsep8pt}
\newcommand{\resumeEntryEnd}{\end{itemize}}

% Itemized list for the bullet points under an entry, if necessary
\newcommand{\resumeItemListStart}{\begin{itemize}[leftmargin=4.5mm]\itemsep-3pt}
\newcommand{\resumeItemListEnd}{\end{itemize}}

% Resume item
\renewcommand{\labelitemii}{\bulletstyle}
\newcommand{\resumeItem}[1]{
  \item\small{
    {#1}
  }
}

% Entry with title, subheading, date(s), and location
\newcommand{\resumeEntryTSDL}[4]{
  \item[]
    \begin{tabularx}{0.97\textwidth}{X@{\hspace{60pt}}r}
      \textbf{\color{primary}#1} & {\firabook\color{accent}\small#2} \\
      \vspace{-0.3cm}
      \textit{\color{accent}\small#3} & \textit{\color{accent}\small#4} \\
    \end{tabularx}\vspace{-0.1cm}
}

% Entry with title and date(s)
\newcommand{\resumeEntryTD}[2]{
  \item[]
    \begin{tabularx}{0.97\textwidth}{X@{\hspace{60pt}}r}
      \textbf{\color{primary}#1} & {\firabook\color{accent}\small#2}
    \end{tabularx}\vspace{3pt}
}

% Entry for special (skills)
\newcommand{\resumeEntryS}[2]{
  \item[]\small{
    \textbf{\color{primary}#1 }{ #2 }
  }
}

% Triple column header
\newcommand{\triplecol}[3]{
	\vspace{-0.3cm}
	\begin{tabularx}{\textwidth}{XXX}
	{\small#1} & {\small#2} & {\small#3}
	\end{tabularx}
}

% Fourth column header
\newcommand{\fourthcol}[4]{
	\vspace{-0.3cm}
	\begin{tabularx}{\textwidth}{XXXX}
		{\small#1} & {\small#2} & {\small#3} & {\small#4}
	\end{tabularx}
}

% Double column header
\newcommand{\doublecol}[6]{
  \begin{tabularx}{\textwidth}{Xr}
    {
      \begin{tabular}[c]{l}
        \fontsize{35}{45}\selectfont{\color{primary}{{\textbf{\fullname}}}} \\
        {\textit{\subtitle}} % You could add a subtitle here
      \end{tabular}
    } & {
      \begin{tabular}[c]{l@{\hspace{1.5em}}l}
        {\small#4} & {\small#1} \\
        {\small#5} & {\small#2} \\
        {\small#6} & {\small#3}
      \end{tabular}
    }
  \end{tabularx}
\vspace{0.3cm}
}

% Single column header
\newcommand{\singlecol}[6]{
  \begin{tabularx}{\textwidth}{Xr}
    {
      \begin{tabular}[b]{l}
        \fontsize{35}{45}\selectfont{\color{primary}{{\textbf{\fullname}}}} \\
        {\textit{\subtitle}} % You could add a subtitle here
      \end{tabular}
    } & {
      \begin{tabular}[c]{l}
        {\small#1} \\
        {\small#2} \\
        {\small#3} \\
        {\small#4} \\
        {\small#5} \\
        {\small#6}
      \end{tabular}
    }
  \end{tabularx}
}

\begin{document}
%-------------------------------------------------- BEGIN HERE --------------------------------------------------

%---------------------------------------------------- HEADER ----------------------------------------------------

\headertype{\linkedin}{\github}{}{\phone}{\email}{} % Set the order of items here

%-------------------------------------------------- EDUCATION --------------------------------------------------
\section{\faGraduationCap}{Education}

\resumeEntryStart
	\small
	\resumeEntryTSDL
    {University of Campinas (UNICAMP)}{São Paulo, Brazil}
	{M.Sc. \& Ph.D. in Physics}{Current}

	\resumeEntryTSDL
	{National University of Engineering (UNI)}{Lima, Peru}
	{B.Sc. in Physics}{2013}
\resumeEntryEnd


%-------------------------------------------------- SUMMARY --------------------------------------------------
\section{\faFolderOpen}{Summary}
\resumeItemListStart
\resumeItem {Experience working with optics and photonics, with emphasis on free-space and fiber optics experiments.}
\resumeItem {Strong capacity to design science experiments and automate sophisticated scientific instruments.}
\resumeItem {Proficient in a variety of specialized computer programs to simulate, acquire, analyze, and visualize data.}
%\resumeItem {Ability to combine quantitative methods of mathematics with applied science in order to solve real problems.}
\resumeItem {Considerable experience teaching students at the undergraduate level.}
\resumeItemListEnd

%-------------------------------------------------- LANGUAGES --------------------------------------------------
\section{\faComment}{Languages}
\resumeEntryStart
	\small
	\triplecol{\resumeEntryS{Spanish}{Native}}{\resumeEntryS{Portuguese}{Fluent}}{\resumeEntryS{English} {Professional}}
\resumeEntryEnd

%-------------------------------------------------- PROJECTS --------------------------------------------------
\section{\faFlask}{Projects}

\resumeEntryStart
\small
\resumeEntryTD
{Dispersion engineering in optical microcavities}{UNICAMP, 2016-2020}
{We engineered the group velocity dispersion (GVD) of silica microresonators using different methods. In wedge microresonator, we did it by controlling its sidewall angle without affecting significantly the free spectral range \href{https://arxiv.org/abs/2003.11625v1}{\faExternalLink}. In spherical microresonator, we used ALD alumina coating of different thicknesses with the same intention \href{https://arxiv.org/abs/2009.07826}{\faExternalLink}. We reported both methods in scientific journals showing the capability of both microresonators to produce tailored optical frequency combs. Here, I wrote scripts in Python that allowed us to build a reliable frequency axis using a calibrated fiber-MZI, recognize all mode families through their quality-factors, and analyze dispersion from dense optical spectra. This project was an excellent opportunity to acquire new knowledge and skills to fabricate silica microresonators, simulate their optical modes, communicate sophisticated equipment from our hardware, process AFM images, analyze millions of data point, and optimize processes using several Python libraries. The data that support the findings of this project are openly available in Zenodo \href{https://zenodo.org/record/3932243}{\faExternalLink}.}

\resumeEntryTD
{Tunable light filters}{SAMSUNG \& UNICAMP, 2017}
{It was a partnership between SAMSUNG and the Device Research Laboratory (LPD-Unicamp) where I participated in contributing to the colour theory transformations and algorithms necessary to identify colors emitted by the homemade filters. For this, we used an spectrometer and a Python \href{https://www.colour-science.org/}{package} for colour science.}

\resumeEntryTD
{High sensitivity spectroscopy}{UNICAMP, 2014-2015}
{In this project, I demonstrated the possibility of using optical cavities of moderate finesse for measurements of small absorption coefficients of nearly transparent liquid and solid samples. With this sensitive technique, based on measurements of ring-down times, I isolated the absorption coefficient of liquids contained inside a transparent cuvette oriented at Brewster's angle. This project served to acquire experience working on spectroscopy and free-space optics. \href{http://repositorio.unicamp.br/handle/REPOSIP/305744}{Link in portuguese \faExternalLink}}

\resumeEntryTD
{Magnetic properties of \ce{CuO2} nanoparticles on graphite and graphene}{UFABC, 2012}
{Here, I focused on obtaining graphene from highly oriented graphite blocks using the scotch tape method. Afterwards, we obtained nanoparticles by laser ablation and deposited on graphene samples. The optical and magnetic characterization of the samples were done with the intention of detect changes in their properties.}

\resumeEntryEnd

%-------------------------------------------------- PUBLICATIONS --------------------------------------------------
\section{\faBarChart}{Publications}
\small
\vspace{10pt}

\textbf{Journals}
\vspace{-5pt}
\resumeItemListStart
\resumeItem {\textbf{M. Inga}, L. Fujii, J. M. da Silva Filho, J. Quintino, A. Ferlauto, F. C. Marques, T. P. M. Alegre, and G. S. Wiederhecker. Alumina coating for dispersion management in ultra-high Q microresonators. Under Review, \href{https://arxiv.org/abs/2009.07826}{arXiv \faExternalLink}.}
\vspace{2pt}
\resumeItem {L. Fujii, \textbf{M. Inga}, J. H. Soares, Y. A. V. Espinel, T. P. Mayer Alegre, and G. S. Wiederhecker. Dispersion tailoring in wedge microcavities for Kerr comb generation. Optics Letters Vol. 45, Issue 12, pp. 3232-3235 (2020). \href{https://www.osapublishing.org/ol/abstract.cfm?uri=ol-45-12-3232}{\faExternalLink}}
\resumeItemListEnd

\textbf{Conferences}
\vspace{-5pt}
\resumeItemListStart
\resumeItem {\textbf{M. Inga}, L. F. dos Santos, J. M. C. da Silva Filho, Y. A. V. Espinel, F. C. Marques, T. P. M. Alegre, and G. S. Wiederhecker. Tailoring group-velocity dispersion in microspheres with alumina coating. In CLEO, pp JTh2C.4. Optical Society of America (2020). \href{https://www.osapublishing.org/abstract.cfm?uri=CLEO_AT-2020-JTh2C.4}{\faExternalLink}}
\vspace{2pt}
\resumeItem{L. Fujii, \textbf{M. Inga}, J. H. Soares, T. P. Mayer Alegre, and G. S. Wiederhecker. Dispersion Control in Silicon Oxide Wedge Microdisks. In CLEO: QELS Fundamental Science, pp. JTu2A-111. Optical Society of America (2018). \href{https://www.osapublishing.org/abstract.cfm?uri=CLEO_SI-2018-JTu2A.111}{\faExternalLink}}
\resumeItemListEnd

%-------------------------------------------------- Teacher Internship Programs --------------------------------------------------
\section{\faPencil}{Teaching Experience}
\resumeEntryStart
\resumeEntryTD
{\small Electric Circuits and Electromagnetism}{UNIVESP, 2019-II} 
\small{Employed on a temporary contract by the UNIVESP in teaching-related responsibilities.}
\resumeEntryTD
{Experimental Physics IV: Alternating Current and Optics}{UNICAMP, 2015-II, 2016-II}
\small{Participating in the Docent Training Stage Program at UNICAMP.}
\resumeEntryTD
{Experimental Physics III: Electricity and Magnetism}{UNICAMP, 2014-II, 2016-I}
\small{Participating in the Docent Training Stage Program at UNICAMP.}
\resumeEntryEnd

%-------------------------------------------------- Professional Affiliations --------------------------------------------------
\section{\faGroup}{Professional Affiliation}
\resumeEntryStart
\resumeEntryTD
{\small Optical Society of America (OSA)}{2007-Current}
\small{Founder member of the OSA Student Chapter UNI, Lima, Peru. Currently, as a member of the OSA Student Chapter UNICAMP, São Paulo, Brazil.}
\resumeEntryEnd

%-------------------------------------------------- Computer skills --------------------------------------------------
\section{\faDesktop}{Computer skills}
\resumeItemListStart
\resumeItem{I use Latex for scientific reports and Mendeley as a reference manager. To create and edit vectorial images, I use Inkscape. Frequently, my presentations have been done in Impress, PowerPoint or Beamer.}
\resumeItem{Because Office 365 is free for academic institutions, I usually use Microsoft Teams for communication and collaboration.}
\resumeItem{I use pyVISA and pyQt to control instruments and automate experiments. For data exploration and visualization, I use Jupyter notebooks, Pandas, Numpy, Scipy, Sympy, and Matplotlib. Specific problems always will require the use of specialized Python packages.}
\resumeItem{I use Linux as a development and production environment, but sometimes in the lab, Windows is required.}
\resumeItem{I have a strong preference for open-source software, but if I have access to competitive proprietary software like Comsol, Mathematica or Matlab, I will be able to use them too.}
\resumeItemListEnd

%-------------------------------------------------- Interests --------------------------------------------------
\section{\faThumbTack}{Interests}
\resumeEntryStart
\small
\fourthcol{\resumeEntryS{}{Microcomb technology}}{\resumeEntryS{}{Optical sensing}}{\resumeEntryS{}{Optical spectroscopy}}{\resumeEntryS{}{Biophotonics}}
\resumeEntryEnd

%-------------------------------------------------- References --------------------------------------------------
\section{\faTripadvisor}{References}

\vspace{5pt}
Two professors with whom I have worked very closely in my graduate and postgraduate studies are:
\begin{description}
	\item[Dr. Carmen Eyzaguirre] Professor at \href{https://fc.uni.edu.pe/fc/index.php/noticias-secundarias/item/12-eyzaguirre-gorvenia-carmen}{Optics and Photonics Laboratory} - UNI. \href{ceyzaguirre@uni.edu.pe}{\faEnvelopeO} ceyzaguirre@uni.edu.pe
	\item[Dr. Gustavo Wiederhecker] Professor at \href{https://sites.ifi.unicamp.br/lpd/}{Device Research Laboratory} - UNICAMP. \href{gsw@unicamp.br}{\faEnvelopeO} gsw@unicamp.br
\end{description}

\end{document}