%-------------------------------------
% LaTeX Resume for Physicist
% Author : Marvyn Inga
%-------------------------------------

\documentclass[letterpaper, 11pt]{article}[leftmargin=*]

\usepackage[empty]{fullpage}
\usepackage{enumitem}
\usepackage{ifxetex}
\ifxetex
  \usepackage{fontspec}
  \usepackage[xetex]{hyperref}
\else
  \usepackage[utf8]{inputenc}
  \usepackage[T1]{fontenc}
  \usepackage[pdftex]{hyperref}
\fi
\usepackage{fontawesome}
\usepackage[sfdefault,light]{FiraSans}
\usepackage{anyfontsize}
\usepackage{xcolor}
\usepackage{tabularx}
\usepackage{pdflscape}
\usepackage{mhchem}

%-------------------------------------------------- SETTINGS HERE --------------------------------------------------
% Header settings
\def \fullname {\Huge Marvyn Inga}
\def \subtitle {Physicist, \faMale}

\def \linkedinicon {\faLinkedin}
\def \linkedinlink {https://www.linkedin.com/in/marvync/}
\def \linkedintext {/marvync}

\def \phoneicon {\faPhone}
\def \phonetext {+55-21-975106507}

\def \emailicon {\faEnvelope}
\def \emaillink {mailto:marvyn.inga@gmail.com}
\def \emailtext {marvyn.inga@gmail.com}

\def \githubicon {\faGithub}
\def \githublink {https://github.com/marvync}
\def \githubtext {/marvync}

\def \headertype {\doublecol} % \singlecol or \doublecol

% Misc settings
\def \bulletstyle {\faAngleRight}

% Define colours
\definecolor{primary}{HTML}{414141}
\definecolor{secondary}{HTML}{DC3522}
\definecolor{accent}{HTML}{414141}
\definecolor{links}{HTML}{DC3522}
\definecolor{boxcol}{HTML}{E8E8E8}

%------------------------------------------------------------------------------------------------------------------- 

% Defines to make listing easier
\def \linkedin {\linkedinicon \hspace{3pt}\href{\linkedinlink}{\linkedintext}}
\def \phone {\phoneicon \hspace{3pt}{ \phonetext}}
\def \email {\emailicon \hspace{3pt}\href{\emaillink}{\emailtext}}
\def \github {\githubicon \hspace{3pt}\href{\githublink}{\githubtext}}
\def \website {\websiteicon \hspace{3pt}\href{\websitelink}{\websitetext}}

% Adjust margins
\addtolength{\oddsidemargin}{-0.55in}
\addtolength{\evensidemargin}{-0.55in}
\addtolength{\textwidth}{1.1in}
\addtolength{\topmargin}{-0.6in}
\addtolength{\textheight}{1.1in}

% Define the link colours
\hypersetup{
    colorlinks=true,
    urlcolor=links,
}

% Set the margin alignment 
\raggedbottom
\raggedright
\setlength{\tabcolsep}{0in}

%-------------------------
% Custom commands

% Sections
\renewcommand{\section}[2]{
  \colorbox{boxcol}{\color{secondary}\raggedbottom\normalsize{#1}{\hspace{2pt}#2}}
}

% Entry start and end, for spacing
\newcommand{\resumeEntryStart}{\begin{itemize}[leftmargin=2.5mm]\itemsep8pt}
\newcommand{\resumeEntryEnd}{\end{itemize}}

% Itemized list for the bullet points under an entry, if necessary
\newcommand{\resumeItemListStart}{\begin{itemize}[leftmargin=4.5mm]\itemsep-3pt}
\newcommand{\resumeItemListEnd}{\end{itemize}}

% Resume item
\renewcommand{\labelitemii}{\bulletstyle}
\newcommand{\resumeItem}[1]{
  \item\small{
    {#1}
  }
}

% Entry with title, subheading, date(s), and location
\newcommand{\resumeEntryTSDL}[4]{
  \item[]
    \begin{tabularx}{0.98\textwidth}{X@{\hspace{60pt}}r}
      \textbf{\color{primary}#1} & {\firabook\color{accent}\small#2} \\
      \vspace{-0.35cm}
      \textit{\color{accent}\small#3} & \textit{\color{accent}\small#4} \\
    \end{tabularx}\vspace{-0.35cm}
}

% Entry with title and date(s)
\newcommand{\resumeEntryTD}[2]{
  \item[]
    \begin{tabularx}{0.97\textwidth}{X@{\hspace{60pt}}r}
      \textbf{\color{primary}#1} & {\firabook\color{accent}\small#2}
    \end{tabularx}\vspace{2pt}
}

% Entry for special (skills)
\newcommand{\resumeEntryS}[2]{
  \item[]\small{
    \textbf{\color{primary}#1 }{ #2 }
  }
}

% Triple column header
\newcommand{\triplecol}[3]{
	\vspace{-0.3cm}
	\begin{tabularx}{\textwidth}{XXX}
	{\small#1} & {\small#2} & {\small#3}
	\end{tabularx}
}

% Fourth column header
\newcommand{\fourthcol}[4]{
	\vspace{-0.3cm}
	\begin{tabularx}{\textwidth}{XXXX}
		{\small#1} & {\small#2} & {\small#3} & {\small#4}
	\end{tabularx}
}

% Double column header
\newcommand{\doublecol}[6]{
  \begin{tabularx}{\textwidth}{Xr}
    {
      \begin{tabular}[c]{l}
        \fontsize{35}{45}\selectfont{\color{primary}{{\textbf{\fullname}}}} \\
        {\textit{\subtitle}} % You could add a subtitle here
      \end{tabular}
    } & {
      \begin{tabular}[c]{l@{\hspace{1.5em}}l}
        {\small#4} & {\small#1} \\
        {\small#5} & {\small#2} \\
        {\small#6} & {\small#3}
      \end{tabular}
    }
  \end{tabularx}
\vspace{0.3cm}
}

% Single column header
\newcommand{\singlecol}[6]{
  \begin{tabularx}{\textwidth}{Xr}
    {
      \begin{tabular}[b]{l}
        \fontsize{35}{45}\selectfont{\color{primary}{{\textbf{\fullname}}}} \\
        {\textit{\subtitle}} % You could add a subtitle here
      \end{tabular}
    } & {
      \begin{tabular}[c]{l}
        {\small#1} \\
        {\small#2} \\
        {\small#3} \\
        {\small#4} \\
        {\small#5} \\
        {\small#6}
      \end{tabular}
    }
  \end{tabularx}
}

\begin{document}
%-------------------------------------------------- BEGIN HERE --------------------------------------------------

%---------------------------------------------------- HEADER ----------------------------------------------------

\headertype{\linkedin}{\github}{}{\phone}{\email}{} % Set the order of items here

%-------------------------------------------------- EDUCATION --------------------------------------------------
\section{\faGraduationCap}{Education}
\resumeEntryStart
	\small
	\resumeEntryTSDL
    {Ph.D. in Physics}{São Paulo, Brazil}
	{\footnotesize University of Campinas - UNICAMP}{\footnotesize 2015 - Current}
	\resumeEntryTSDL
	{M.Sc. in Physics}{São Paulo, Brazil}
	{\footnotesize University of Campinas - UNICAMP}{\footnotesize 2013 - 2015}
	\resumeEntryTSDL
	{B.Sc. in Physics}{Lima, Peru}
	{\footnotesize National University of Engineering - UNI}{\footnotesize 2006 - 2012}
\resumeEntryEnd
\vspace{0.1cm}
%-------------------------------------------------- SUMMARY --------------------------------------------------
\section{\faFolderOpen}{Summary}
\resumeItemListStart
\resumeItem {Experience working with optics and photonics, with emphasis on the experimental work.}
\resumeItem {Strong capacity to design science experiments and automate sophisticated instruments.}
\resumeItem {Proficient in a variety of specialized computer programs to acquire, analyze, and visualize data.}
%\resumeItem {Ability to combine quantitative methods of mathematics with applied science in order to solve real problems.}
\resumeItem {Considerable experience teaching students at the undergraduate level.}
\resumeItemListEnd

%-------------------------------------------------- LANGUAGES --------------------------------------------------
\section{\faComment}{Languages}
\vspace{-0.2cm}
\resumeEntryStart
	\small
	\triplecol{\resumeEntryS{Spanish}{Native}}{\resumeEntryS{Portuguese}{Fluent}}{\resumeEntryS{English} {Professional}}
\resumeEntryEnd

%-------------------------------------------------- PROJECTS --------------------------------------------------
\section{\faFlask}{Projects}

\resumeEntryStart
\small
\resumeEntryTD
{Kerr optical frequency comb generation in silica microresonators}{\footnotesize UNICAMP, 2016-2020}
{\footnotesize Optical frequency combs (OFCs) are energy-efficient light sources consisting of a series of discrete equally spaced lines in the frequency domain. OFCs can be used for frequency metrology, precision spectroscopy, distance measurement or telecommunications, just to name a few applications. In this project, we engineered the group velocity dispersion of silica microresonators with the intention to generate more suitable OFCs via the Kerr nonlinearity. For example, in wedge microresonator we did it by controlling its sidewall angle without affecting significantly the free spectral range. In spherical microresonator, we used ALD alumina coatings of different thicknesses with the same intention. We published our findings in scientific journals like \href{https://www.osapublishing.org/ol/abstract.cfm?uri=ol-45-12-3232}{Optics Letters} and \href{https://aip.scitation.org/doi/10.1063/5.0028839}{APL Photonics}. My contribution to this project was principally in the experimental work, automating instruments, acquiring and analyzing data. Among others things, I wrote Python scripts with different objectives like identifying and characterizing all mode families through their optical properties (e.g., quality-factor or extinction ratio) or analyzing the dispersion of the microresonators from dense optical spectra.}

\resumeEntryTD
{Tunable light filters}{\footnotesize SAMSUNG \& UNICAMP, 2017}
{\footnotesize It was a partnership between SAMSUNG and the Device Research Laboratory (LPD-Unicamp) where I participated in contributing to the colour theory transformations and algorithms necessary to identify colors emitted by the homemade filters. For this, I used an spectrometer and a Python \href{https://www.colour-science.org/}{package} for colour science.}

\resumeEntryTD
{High sensitivity spectroscopy}{\footnotesize UNICAMP, 2014-2015}
{\footnotesize In this project, I demonstrated the possibility of using optical cavities of moderate finesse for measurements of small absorption coefficients of nearly transparent liquid and solid samples. With this sensitive technique, based on measurements of ring-down times, I isolated the absorption coefficient of liquids contained inside a transparent cuvette oriented at Brewster's angle. This project was important to acquire experience working on spectroscopy and free-space optics. \href{http://repositorio.unicamp.br/handle/REPOSIP/305744}{Link in portuguese \faExternalLink}}

\resumeEntryTD
{Magnetic properties of \ce{CuO2} nanoparticles on graphite and graphene}{\footnotesize UFABC, 2012}
{\footnotesize Here, I focused on obtaining graphene from highly oriented graphite blocks using the scotch tape method. Afterwards, we obtained nanoparticles by laser ablation and deposited on graphene samples. The optical and magnetic characterization of the samples were done with the intention of detect changes in their properties.}

\resumeEntryEnd

%-------------------------------------------------- PUBLICATIONS --------------------------------------------------
\section{\faBarChart}{Publications}
\small
\vspace{6pt}

\textbf{Journals}
\vspace{-3pt}
\resumeItemListStart
\resumeItem {\footnotesize \underline{M. Inga}, L. Fujii, J. M. da Silva Filho, J. Quintino, A. Ferlauto, F. C. Marques, T. P. M. Alegre, and G. S. Wiederhecker. Alumina coating for dispersion management in ultra-high Q microresonators. \href{https://aip.scitation.org/doi/10.1063/5.0028839}{\textit{APL Photonics}} 5, 116107 (2020). This article was chose by the editors as a Featured Article. \href{https://aip.scitation.org/doi/10.1063/5.0028839}{\faExternalLink}}
\vspace{2pt}
\resumeItem {\footnotesize L. Fujii, \underline{M. Inga}, J. H. Soares, Y. A. V. Espinel, T. P. Mayer Alegre, and G. S. Wiederhecker. Dispersion tailoring in wedge microcavities for Kerr comb generation. \href{https://www.osapublishing.org/ol/abstract.cfm?uri=ol-45-12-3232}{\textit{Optics Letters}} Vol. 45, Issue 12, pp. 3232-3235 (2020). \href{https://www.osapublishing.org/ol/abstract.cfm?uri=ol-45-12-3232}{\faExternalLink}}
\resumeItemListEnd

\textbf{Conferences}
\vspace{-3pt}
\resumeItemListStart
\resumeItem {\footnotesize \underline{M. Inga}, L. F. dos Santos, J. M. C. da Silva Filho, Y. A. V. Espinel, F. C. Marques, T. P. M. Alegre, and G. S. Wiederhecker. Tailoring group-velocity dispersion in microspheres with alumina coating. In \href{https://www.osapublishing.org/abstract.cfm?uri=CLEO_AT-2020-JTh2C.4}{\textit{CLEO}}, pp JTh2C.4. Optical Society of America (2020). \href{https://www.osapublishing.org/abstract.cfm?uri=CLEO_AT-2020-JTh2C.4}{\faExternalLink}}
\vspace{3pt}
\resumeItem{\footnotesize L. Fujii, \underline{M. Inga}, J. H. Soares, T. P. Mayer Alegre, and G. S. Wiederhecker. Dispersion Control in Silicon Oxide Wedge Microdisks. In \href{https://www.osapublishing.org/abstract.cfm?uri=CLEO_SI-2018-JTu2A.111}{\textit{CLEO}}, pp. JTu2A-111. Optical Society of America (2018). \href{https://www.osapublishing.org/abstract.cfm?uri=CLEO_SI-2018-JTu2A.111}{\faExternalLink}}
\resumeItemListEnd

\textbf{Sharing and curation of data and software}
\vspace{-3pt}
\resumeItemListStart
\resumeItem {\footnotesize \underline{M. Inga}, L. Fujii, J. M. da Silva Filho, J. Quintino, A. Ferlauto, F. C. Marques, T. P. M. Alegre, and G. S. Wiederhecker. (2020). Dataset and Simulation Files for article "Alumina coating for dispersion management in ultra-high Q microresonators" (Data set v1.0) \href{https://zenodo.org/record/3932243}{Zenodo \faExternalLink}.}
\resumeItemListEnd

%-------------------------------------------------- Teacher Internship Programs --------------------------------------------------
\section{\faPencil}{Teaching Experience}
\resumeEntryStart
\resumeEntryTD
{Electric Circuits and Electromagnetism}{\footnotesize UNIVESP, 2019-II\vspace{-0.2cm}} 
{\footnotesize Employed on a temporary contract by the UNIVESP in teaching-related responsibilities.}
\vspace{-0.3cm}
\resumeEntryTD
{Experimental Physics IV: Alternating Current and Optics}{\footnotesize UNICAMP, 2015-II, 2016-II\vspace{-0.2cm}}
{\footnotesize Participating in the Docent Training Stage Program at UNICAMP.}
\vspace{-0.3cm}
\resumeEntryTD
{Experimental Physics III: Electricity and Magnetism}{\footnotesize UNICAMP, 2014-II, 2016-I\vspace{-0.2cm}}
{\footnotesize Participating in the Docent Training Stage Program at UNICAMP.}
\resumeEntryEnd

%-------------------------------------------------- Professional Affiliations --------------------------------------------------
\section{\faGroup}{Professional Affiliation}
\resumeEntryStart
\resumeEntryTD
{Optical Society of America (OSA)}{\footnotesize 2007-Current}
{\footnotesize Founder member of the OSA Student Chapter UNI, Lima, Peru. Currently, as a member of the OSA Student Chapter UNICAMP, SP, Brazil.}
\resumeEntryEnd

%-------------------------------------------------- Computer skills --------------------------------------------------
\section{\faDesktop}{Computer skills}
\resumeItemListStart
\resumeItem{\footnotesize I use Latex or Google Docs for scientific reports and Mendeley as a reference manager. For making scientific illustrations for journal articles, I use Inkscape or Blender. Frequently, my presentations have been done in Impress or Google Slides.}
\resumeItem{\footnotesize I use pyVISA and pyQt to control instruments and automate experiments. For data exploration and visualization, I use Jupyter notebooks, Pandas, Numpy, Scipy, Sympy, and Matplotlib.}
\resumeItem{\footnotesize I use preferentially Linux as a development and production environment.}
\resumeItem{\footnotesize I have a strong preference for open-source software, but if I have access to competitive proprietary software like Comsol, Mathematica or Matlab, I will be able to use them too.}
\resumeItem{\footnotesize I usually use Microsoft Teams for communication and collaboration.}
\resumeItemListEnd

%-------------------------------------------------- Certifications --------------------------------------------------
\section{\faCertificate}{Certifications}

\vspace{0.2cm}
{\footnotesize I am continually updating and improving my programming skills to meet new challenges and changing needs.}
\vspace{-0.1cm}
\resumeItemListStart
\resumeItem{\textbf{Python for Data Science}\par
\vspace{-0.15cm}
{\footnotesize Issued by IBM on 2020. The badge earner is able to write their own Python scripts and perform basic hands-on data analysis using IBM's Jupyter-based lab environment. \href{https://www.youracclaim.com/badges/0f8785bb-1050-4da5-8dd3-15d8bf1d68a8}{See credential \faExternalLink}}}
\vspace{0.1cm}
\resumeItem{\textbf{Data Analysis Using Python}\par
\vspace{-0.15cm}
{\footnotesize Issued by IBM on 2020. This badge earner understands the essential steps necessary to analyze data in Python using multi-dimensional arrays, manipulating DataFrames in pandas, using SciPy library of mathematical routines and performing machine learning using scikit-learn. \href{https://www.youracclaim.com/badges/6254609b-3350-47b3-83e9-d7e6ff2a1daa}{See credential \faExternalLink}}}
\vspace{0.1cm}
\resumeItem{\textbf{Data Visualization Using Python}\par
\vspace{-0.15cm}
{\footnotesize Issued by IBM on 2020. This badge earner understands how Python libraries such as Matplotib, Seaborn and Folium are used for the creation and customization of graphical representation outputs for both small and large-scale data sets. \href{https://www.youracclaim.com/badges/6e5beb76-93e0-4f62-9a4d-6de24671215f}{See credential \faExternalLink}}}
\vspace{0.1cm}
\resumeItem{\textbf{Applied Data Science with Python}\par
\vspace{-0.15cm}
{\footnotesize Issued by IBM on 2020. This badge earner is able to code in Python for data science. They can analyze and visualize data with Python with packages like scikit-learn, matplotlib and bokeh. \href{https://www.youracclaim.com/badges/fb8ee1fc-0a04-4027-ad0e-21179858d327}{See credential \faExternalLink}}}
\resumeItemListEnd

%-------------------------------------------------- Interests --------------------------------------------------
\section{\faThumbTack}{Interests}
\vspace{-0.6cm}
\resumeEntryStart
\footnotesize
\fourthcol{\resumeEntryS{}{\item \footnotesize Microcomb technology}}{\resumeEntryS{}{\item \footnotesize Optical spectroscopy}}{\resumeEntryS{}{\item \footnotesize Optical sensing}}{\resumeEntryS{}{\item \footnotesize Biophotonics}}
\vspace{-0.7cm}
\fourthcol{\resumeEntryS{}{\item \footnotesize Data Analyst}}{\resumeEntryS{}{\item \footnotesize Artificial Intelligence}}{\resumeEntryS{}{\item \footnotesize 5G Technology}}{\resumeEntryS{}{\item \footnotesize Digital innovation}}
\resumeEntryEnd

%-------------------------------------------------- References --------------------------------------------------
\section{\faTripadvisor}{References}

\footnotesize
\vspace{4pt}
Two professors with whom I worked very closely in my graduate and postgraduate studies are, respectively,
\begin{description}
	\item[Dr. Carmen Eyzaguirre] Professor at \href{https://fc.uni.edu.pe/fc/index.php/noticias-secundarias/item/12-eyzaguirre-gorvenia-carmen}{Optics and Photonics Laboratory} - UNI. \href{ceyzaguirre@uni.edu.pe}{\faEnvelopeO} ceyzaguirre@uni.edu.pe
	\vspace{-0.05cm}
	\item[Dr. Gustavo Wiederhecker] Professor at \href{https://sites.ifi.unicamp.br/lpd/}{Device Research Laboratory} - UNICAMP. \href{gsw@unicamp.br}{\faEnvelopeO} gsw@unicamp.br
\end{description}

\end{document}